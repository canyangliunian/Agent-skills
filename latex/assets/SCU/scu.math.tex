%-------------------------------------------------------------
% 导言区
\documentclass{beamer}
\usepackage[UTF8]{ctex}
\usepackage{hyperref}
\usepackage[T1]{fontenc}

\usepackage{latexsym,xcolor,multicol,booktabs,calligra}
\usepackage{amsmath,amssymb,BOONDOX-cal,bm}		%   数学公式宏包
\usepackage{graphicx,pstricks,stackengine}      %   请自行查找相关宏包作用

%   个人信息
\author{mountain}
\title{SCU Beamer Theme}
\subtitle{毕 业 论 文 开 题 报 告}
\institute{四 川 大 学 数 学 学 院}     % 在这可以修改学院
\date{\today}

%   右下角添加川大logo 你也可以使用eps或jpg格式 
%   下面是两种添加logo的code 不同的文件有不同的清晰度 自行选择合适的code
%\pgfdeclareimage[height=0.8cm]{SCU-Logo}{SCU-Logo.pdf}  
%\logo{\pgfuseimage{SCU-Logo}\hspace*{0.3cm}}
\logo{\includegraphics[height=0.8cm]{SCU-Logo.jpg}\hspace*{0.3cm}}

%   设置主题文件    后缀名为.sty的文件是一个主题文件,初学者不要修改sty文件
\usepackage{SCU}



%   defs 特殊字体
\def\cmd#1{\texttt{\color{red}\footnotesize $\backslash$#1}}
\def\env#1{\texttt{\color{blue}\footnotesize #1}}


%   简化定理命令
\newtheorem{thm}{Theorem}[theorem]


%—-------------------------------------------------------------
% 正文区

\begin{document}
    % 字体字号设置
    \kaishu \zihao{-5}
	
	% 封面
	\begin{frame}
		\titlepage
		%   校徽    在陈老师的建议下封面不添加logo,只在所有页面右下角添加川大Logo
		%\begin{figure}[htpb]
		%	\begin{center}
		%		\includegraphics[width=0.2\linewidth]{SCU_Logo.png} 
		%	\end{center}
		%\end{figure}
	\end{frame}
	
	% 目录
	\begin{frame}
		\tableofcontents[sectionstyle=show,subsectionstyle=show/shaded/hide,subsubsectionstyle=show/shaded/hide]
	\end{frame}
		
	
	%—------------------------------------------------------
	% 正文
	\section{课题背景}
	
		\begin{frame}
			\begin{itemize}[<+-| alert@+>] 
	    % structure和alert命令则用于在指定的步骤设置高亮,前者使用幻灯片结构的色彩,后者使用的是更鲜明的警告色彩
				\item 大家都会\LaTeX{},好多学校都有自己的Beamer主题, 但是overleaf上没有川大的模板
				\item 中文支持请选择 Xe\LaTeX{} 编译选项 \qquad 如果你做了一些改动上传overleaf一定要记得菜单栏里选择编译环境为 Xe\LaTeX{}(血泪教训)
				\item Overleaf项目地址位于 \url{https://www.overleaf.com/latex/templates/},可以直接使用
			\end{itemize}
		\end{frame}
	
	%-------------------------------------------------------	
	\section{研究现状}
	    \subsection{Beamer主题分类}	
	    
		\begin{frame}
			\begin{enumerate}
				\item<1-> 有一些 \LaTeX{} 自带的
				\item<2-> 有一些Tsinghua的
				\item<3-> 本模板基于overleaf上清华大学beamer主题模板 \newline \url{https://cn.overleaf.com/latex/templates/thu-beamer-theme/vwnqmzndvwyb}
				\item<4-> 本模板还参考了南方科技大学beamer主题模板 
				\newline
				\url{https://www.latexstudio.net/archives/11443.html}
			\end{enumerate}	
		\end{frame}
	
	%-----------------------------------------------------
	\section{研究内容}
	    \subsection{美化主题}
	
	\begin{frame}{这一份主题与Overleaf的THU Beamer Theme区别在于}
		\begin{itemize}[<+->]
			\item 右下角添加了川大的logo
			\item 模板中的公式做了一部分的替换
			\item 删除了原版中lstlisting的相关内容,如有需要请转至 \\
			\url{https://cn.overleaf.com/latex/templates/thu-beamer-theme/vwnqmzndvwybl}
			\item 参考文献格式的一些改动
		\end{itemize}
	\end{frame}
		
	\begin{frame}{学习\LaTeX{} 的一些途径}
		\begin{enumerate}[<+->]
			\item  B站耿楠教授的视频 \\ \url{https://www.bilibili.com/video/BV15b411j7Au?spm_id_from=333.999.0.0}
			\item  在Windows 命令提示符或者Linux 终端中查找帮助文档,如CTEX宏集手册、《LaTeX2$\epsilon$完全学习手册(第二版)》等
			\item  \LaTeX{} 科技排版工作室 
			\item  CSDN
			\item  GitHub
		\end{enumerate}
	\end{frame}
	
	\subsection{如何更好地做Beamer}
	
	\begin{frame}
    	\begin{itemize}
    	    \item \LaTeX{} 广泛用于学术界,期刊会议论文模板
    	\end{itemize}
        \begin{table}[h]
    		\centering
    		\begin{tabular}{c|c}
    			Microsoft Word & \LaTeX{} \\
    			\hline
    			文字处理工具 & 专业排版软件 \\
    			容易上手,简单直观 & 容易上手 \\
    			所见即所得 & 所见即所想,所想即所得 \\
    			高级功能不易掌握 & 进阶难,但一般用不到 \\
    			处理长文档需要丰富经验 & 和短文档处理基本无异 \\
    			花费大量时间调格式 & 无需担心格式,专心作者内容 \\
    			公式排版差强人意 & 尤其擅长公式排版 \\
    			二进制格式,兼容性差 & 文本文件,易读、稳定 \\
    			付费商业许可 & 自由免费使用 \\
    		\end{tabular}
    	\end{table}
    \end{frame}
	
    \begin{frame}{排版举例}
	    \begin{exampleblock}{无编号公式} % 加 * 
    		\begin{equation*}
    		f(x+\xi)=\sum_{n=0}^{N}\frac{1}{n!}\xi^n\frac{d^nf(x)}{dx^n}+R(N,x) 
    		\label{equ-1}
    		\end{equation*}
    	\end{exampleblock}
	
    	\begin{exampleblock}{多行多列公式\footnote{\tiny {如果公式中有文字出现,请用 $\backslash$mathrm\{\} 或者 $\backslash$text\{\} 包含,不然就会变成 $clip$,在公式里看起来比 $\mathrm{clip}$ 丑非常多。}}}
    		% 使用 & 分隔
    		\begin{align}
    		    \frac{1}{n!}\int_{H_x}\xi^ng_N^p(\xi)d\xi=\delta_{np}=
    		    \begin{cases}
    			1 & n=p\\
    			0 & n\neq p\\
    		    \end{cases}
    	    	\label{equ-2}
    		\end{align}
    	\end{exampleblock}
    \end{frame}
	
	\begin{frame}
		\begin{exampleblock}{编号多行公式}
			% Taken from Mathmode.tex
			\begin{multline}
			A=\lim_{n\rightarrow\infty}\Delta x\left(a^{2}+\left(a^{2}+2a\Delta x+\left(\Delta x\right)^{2}\right)\right.\label{eq:reset}\\
			+\left(a^{2}+2\cdot2a\Delta x+2^{2}\left(\Delta x\right)^{2}\right)\\
			+\left(a^{2}+2\cdot3a\Delta x+3^{2}\left(\Delta x\right)^{2}\right)\\
			+\ldots\\
			\left.+\left(a^{2}+2\cdot(n-1)a\Delta x+(n-1)^{2}\left(\Delta x\right)^{2}\right)\right)\\
			=\frac{1}{3}\left(b^{3}-a^{3}\right)
			\end{multline}
		\end{exampleblock}
	\end{frame}
	
	\begin{frame}{图片与分栏}
	    % Taken from SUSTech beamer 
		\begin{columns}[T] % align columns
			\begin{column}<0->{.40\textwidth}
				\begin{figure}[thpb]
					\centering
					\resizebox{1\linewidth}{!}{
						\includegraphics{tianyuan.pdf}
					}
					%	\includegraphics[scale=1.0]{figurefile}
					%	对于较大的图片可以转换为pdf格式	
					\caption{\tiny{Tianyuan Mathematical  Center}}
					\label{fig:1}
				\end{figure}
			\end{column}%
			\hfill%
			\begin{column}<0->{.65\textwidth}
				\begin{itemize}
					\item<1-> 国家天元数学西南中心
					\begin{itemize}
						\item<1-> Tianyuan Mathematical Center in Southwest China
					\end{itemize}
					\item<2-> 中华人民共和国四川省成都市
					\begin{itemize}
						\item<2-> Chengdu, Sichuang Province, P.R. China
					\end{itemize}
				\end{itemize}
			\end{column}%
		\end{columns}
	\end{frame}

	\begin{frame}[fragile]{\LaTeX{} 常用命令}
		\begin{exampleblock}{命令}
			\centering
			\footnotesize
			\begin{tabular}{llll}
				\cmd{chapter} & \cmd{section} & \cmd{subsection} & \cmd{paragraph} \\
				章 & 节 & 小节 & 带题头段落 \\ \hline
				\cmd{centering} & \cmd{emph} & \cmd{verb} & \cmd{url} \\
				居中对齐 & 强调 & 原样输出 & 超链接 \\ \hline
				\cmd{footnote} & \cmd{item} & \cmd{caption} & \cmd{includegraphics}  \\
				脚注 & 列表条目 & 标题 & 插入图片 \\ \hline
				\cmd{tabular} & \cmd{label} & \cmd{cite} & \cmd{ref} \\
				插入表格 & 标号 & 引用参考文献 & 引用图表公式等 \\ \hline
			\end{tabular}
		\end{exampleblock}
		\begin{exampleblock}{环境}
			\centering
			\footnotesize
			\begin{tabular}{lll}
				\env{table} & \env{figure} & \env{equation}\\
				表格 & 图片 & 公式 \\\hline
				\env{itemize} & \env{enumerate} & \env{description}\\
				无编号列表 & 编号列表 & 描述 \\\hline
			\end{tabular}
		\end{exampleblock}
	\end{frame}
	
	%-------------------------------------------------
	\section{研究思路}	
	
	\begin{frame}<1->
		 \begin{thm}
	        contents...
	        \label{thm-1}
		 \end{thm}
		%在命令或环境后加覆盖指示: <范围>, 如<1-3>, <2>, \begin{thm}<2->这个定理从第2幅幻灯片开始显示,直到同一帧的幻灯片全部显示完毕.
		\begin{block}{定理1.1}<2->
            This is a theorem.
            \begin{equation}
                a^2 + b^2 = c^2
            	\notag	% 不显示编号
            	\label{equ-3}
            	\end{equation}
        \end{block}
    	%在beamer中,已经预定义了许多定理类环境:theorem、corollary、definition,definitions,fact,example以及examples,它们都以英文名称给出,例如proof环境名称就是“proof”
    	
		\begin{proof}<3->
            \textit{Trivial.} 
        \end{proof}
		            
        \begin{corollary}<4->
            This is a corollay.
            \begin{equation}
                c^2 = b^2 + a^2
                \label{equ-4}
            \end{equation}
        \end{corollary}
	\end{frame}
	
	\begin{frame}{作图}
		\begin{itemize}
			\item 矢量图 eps, ps, pdf
			\begin{itemize}
				\item METAPOST, pstricks, pgf $\ldots$
				\item Xfig, Dia, Visio, Inkscape $\ldots$
				\item Matlab / Excel 等保存为 pdf
			\end{itemize}
			\item 标量图 png, jpg, tiff $\ldots$
			\begin{itemize}
				\item 提高清晰度,避免发虚
				\item 应尽量避免使用
			\end{itemize}
		\item 官网上找到的川大校徽转换为eps出现了变形,所以使用的是jpg格式
		\end{itemize}
		
	\end{frame}

	%--------------------------------------------------
	\section{进度安排}	
	
		\begin{frame}
			\begin{itemize}
				\item 2021.12.29前 录入开题报告内容和开题报告会议安排
				\item 2022.01.06前 开题
				\item 2022.03.24-2022.03.30 提交论文中期检查表
				\item 2022.04.14-2022.04-28 提交初稿
				\item 2022.05.05前 完成论文
				\item 2022.05.05-2022.05.10 在线提交答辩初稿,完成论文第一次查重
				\item 2022.05.09-2022.05.11 完成论文第二次查重,提交答辩安排
				\item 2022.05.12-2022.05.18 完成论文答辩
			\end{itemize}
			在这引入一个参考文献 \cite{name1}	
		\end{frame}
	
	%----------------------------------------------
	\section{参考文献}
		
		\begin{frame}[allowframebreaks]
			
			\bibliographystyle{plain}
			\bibliography{ref}	
			% 如果参考文献太多的话,可以像下面这样调整字体:
			% \tiny\bibliographystyle{alpha}
            
			关于Bib Tex的更多内容请自行查找
		\end{frame}

	%-------------------------------------------
	\begin{frame}
		\begin{center}
			{\Huge \emph {\textrm{Thank  ~you!}}}
		\end{center}
	\end{frame}

\end{document}